%%%%%%%%%%%%%%%%%%%%%% 
% SYMBOLS            % 
%%%%%%%%%%%%%%%%%%%%%% 

% \newcommand{\bsp}[1]{\textit{#1}} % Probably defines elsewhere
\newcommand{\am}{\parallel}
\newcommand{\unify}{\sqcup}
\newcommand{\word}{\overline}
\newcommand{\Nonterminals}{\mathcal{N}}
\newcommand{\Terminals}{\Sigma}
\newcommand{\Productions}{\mathcal{P}}

\newcommand{\svar}[1]
{\setbox2=\hbox{$\scriptstyle #1$}\lower.2ex\vbox{\hrule
    \hbox{\vrule\kern1.25pt 
      \vbox{\kern1.25pt\box2\kern1.25pt}\kern1.25pt\vrule}\hrule}}
\newcommand{\ssvar}[1]
{\setbox2=\hbox{\scalebox{.7}{$#1$}}\lower.3ex\vbox{\hrule
    \hbox{\vrule\kern1pt 
      \vbox{\kern1pt\box2\kern1pt}\kern1pt\vrule}\hrule}}
\newcommand{\anvar}{\rule{0.4em}{0.4pt}\,}
\newcommand{\trace}[0]{\raisebox{1pt}{\underline{$~~~$}}}

% % Taken from https://tex.stackexchange.com/a/60253/61499
% \makeatletter
% \newcommand*\myrel@kern[1]{\kern#1\dimexpr\macc@kerna}
% \def\widebar#1{% use \def to optionally overwrite \widebar 
%   \begingroup
%   \def\mathaccent##1##2{%
%     \myrel@kern{0.8}%
%     \overline{\myrel@kern{-0.8}\macc@nucleus\myrel@kern{0.2}}%
%     \myrel@kern{-0.2}%
%   }%
%   \macc@depth\@ne
%   \let\math@bgroup\@empty \let\math@egroup\macc@set@skewchar
%   \mathsurround\z@ \frozen@everymath{\mathgroup\macc@group\relax}%
%   \macc@set@skewchar\relax
%   \let\mathaccentV\macc@nested@a
%   \macc@nested@a\relax111{#1}%
%   \endgroup
% }
% \makeatother

%% Dingbats, i.e. symbolic, decorative glyphs
\RequirePackage{pifont}
\newcommand{\cmark}{\ding{51}}  % checkmark
\newcommand{\xmark}{\ding{55}}  % xmark

%% Instead of \operatorname{argmax}
\DeclareMathOperator*{\argmax}{argmax}

%%%%%%%%%%%%%%%%%%%%%%%%%%%%%%%%%%%% 
% COLORS & HIGHLIGHTING            % 
%%%%%%%%%%%%%%%%%%%%%%%%%%%%%%%%%%%% 

\definecolor{myblue}{rgb}{0,0,0.70}
\definecolor{myred}{rgb}{0.8,0,0}
\definecolor{mydarkgreen}{rgb}{0,0.55,0}

\newcommand{\emphR}[1]{{\color{myred}#1}}

%%%%%%%%%%%%%%%%%%%%%% 
% MISCELLANEOUS      % 
%%%%%%%%%%%%%%%%%%%%%% 

% \newcommand{\circled}[1]{\textcircled{\raisebox{-0.9pt}{#1}}}
\DeclareRobustCommand\circled[1]{\tikz[baseline=(char.base)]{ % works better compared to \newcommand*
    \node[shape=circle,draw,inner sep=.15ex] (char) {\small{#1}};}}
% \usepackage{circledsteps}       % provides the macro \Circled; not included in TeXLive 2018
% \newcommand{circled}[1]{\Circled{#1}}


\newcommand{\prule}[3]{\ensuremath{\begin{array}{c} #1\\ \hline #2\end{array} ~~ #3}}

\newcommand{\minitab}[2][c]{\begin{tabular}{@{}#1@{}}#2\end{tabular}}

\newlength{\thisWidth}
\newcommand*{\hspaceThis}[1]{\settowidth{\thisWidth}{#1}\hspace*{\thisWidth}}
\newcommand*{\hspaceMinusThis}[1]{\settowidth{\thisWidth}{#1}\hspace*{-\thisWidth}}

\newenvironment{changemargin}[2]{%
  \begin{list}{}{%
    \setlength{\topsep}{0pt}%
    \setlength{\leftmargin}{#1}%
    \setlength{\rightmargin}{#2}%
    \setlength{\listparindent}{\parindent}%
    \setlength{\itemindent}{\parindent}%
    \setlength{\parsep}{\parskip}%
  }%
  \item[]}{\end{list}}

% \newtheorem{definition}{Definition}
% \newtheorem{corollary}{Corollary}
% \newtheorem{theorem}{Theorem}

\newcommand\redout{\bgroup\markoverwith{\textcolor{red}{\rule[.4ex]{2pt}{1pt}}}\ULon}

\newcommand{\largerpage}[1][1]{
  \enlargethispage{#1\baselineskip}}
