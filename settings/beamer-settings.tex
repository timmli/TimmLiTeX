%%%%%%%%%%%%%%%%%%%%%%%%
%   PDFLATEX SETTINGS  % 
%%%%%%%%%%%%%%%%%%%%%%%%

%%   Font & Encoding

% \usepackage{libertine} % libertine tends to cause problems, e.g. when using tipa
% \usepackage[libertine]{newtxmath}
\usepackage{times}

\makeatletter

\@ifclassloaded{beamer}{
  \usepackage{libertine} % libertine tends to cause problems, e.g. when using tipa
  \usepackage[libertine]{newtxmath}}{}

\@ifclassloaded{tikzposter}{
  \usepackage{DejaVuSans} 
  \renewcommand*{\familydefault}{\sfdefault}}{}

\makeatother

\usepackage[scaled=0.8]{beramono}  % for monospaced font
\usepackage{microtype}		% micro-typographic aspects of the fonts
\usepackage[T1]{fontenc}	% special fonts, e.g. for German umlaute

%% compatible with BibLaTeX
\usepackage[utf8]{inputenc}
%% incompabtible with BibLaTeX
% \usepackage{ucs}
% \usepackage[utf8x]{inputenc}

%% Language
%\usepackage[german]{babel}
%\usepackage{german}
\usepackage[english]{babel}
\usepackage{iflang}             % for language specific settings

\usepackage{etex} 
\usepackage{graphics}

%%%%%%%%%%%%%%%%%%%%%%
%   TIKZ SETTINGS    % 
%%%%%%%%%%%%%%%%%%%%%%

\usepackage{tikz}
\usepackage{tikz-dependency}

\tikzset{every tree node/.style={align=center,anchor=north}}	% to allow linebreaks
\usetikzlibrary{calc} % for positioning arrows with ($(t.center)-(1,0)$)
\usetikzlibrary{shapes,decorations}
\usetikzlibrary{backgrounds,fit}
\usetikzlibrary{arrows}
\usetikzlibrary{matrix}
\usetikzlibrary{positioning}
\usetikzlibrary{automata}
\usetikzlibrary{tikzmark}

% Define box and box title style (see http://www.texample.net/tikz/examples/boxes-with-text-and-math/)
\tikzstyle{mybox} = [draw=gray, very thick,
    rectangle, rounded corners, inner sep=10pt, inner ysep=17pt,yshift=3pt]
\tikzstyle{fancytitle} =[draw=gray, very thick, fill=white,
    rectangle, rounded corners, inner sep=5pt, inner ysep=5pt]
\tikzstyle{mydouble} = [double distance=1pt]
    
\tikzset{
  %% Define standard arrow tip
  >=stealth',
  %% Define style for boxes
  box/.style={
    rectangle,
    rounded corners,
    draw=black, very thick,
    text width=10em,
    minimum height=2em,
    text centered},
  %% Define arrow style
  arrow/.style={
    ->,
    thick,
    shorten <=2pt,
    shorten >=2pt,},
  %% Beamer overlays (https://tex.stackexchange.com/a/253384/61499)
  invisible/.style={opacity=0,text opacity=0},
  visible/.style={alt=#1{}{invisible}},
  alt/.code args={<#1>#2#3}{%
    \alt<#1>{\pgfkeysalso{#2}}{\pgfkeysalso{#3}}},
  transparent/.style={opacity=0.1,text opacity=0.1},
  opaque/.style={alt=#1{}{transparent}},
  alerted/.style={color=alerted text.fg},
  alert/.style={alt=#1{alerted}{}},
}

\newcommand\centertikz[1]{\tikz[baseline=(current bounding box.center)]{#1}}
\newcommand\tikzcenter{baseline=(current bounding box.center)}
\newcommand\tikztop{baseline=(current bounding box.north)}

\newcommand\tikztreeset[1]{\matrix [matrix of nodes,left delimiter=\{,right delimiter=\}](set){#1};}

%%%%%%%%%%%%%%%%%%%%%%% 
%   FOREST SETTINGS   % 
%%%%%%%%%%%%%%%%%%%%%%% 

\usepackage{forest}

\makeatletter

\@ifpackagelater{forest}{2016/01/01}
{\useforestlibrary{linguistics}%
 \useforestlibrary{edges}}
{}

\@ifpackagelater{forest}{2016/01/01}
{\newcommand{\forestPreamble}{default preamble}} % version >=2 of forest
{\newcommand{\forestPreamble}{.style}} % version <=1 of forest

\makeatother

\forestset{
  \forestPreamble ={
    % .style={ % version <=1 of forest
    % default preamble={ % version >=2 of forest    
		for tree={
			parent anchor=south, 
			child anchor=north,
			% align=center,			% bad: adds space below label
			fit=rectangle,
			base=top,				% vertical orientation of nodes
			% inner sep=3,			% necesssary?
			begin draw/.code={\begin{tikzpicture}[baseline=(current bounding box.center)]},
    }},
  htree/.style={for tree={grow'=east,parent anchor=east,child anchor=west,anchor=base west}},
  sn edges/.style={for tree={parent anchor=south, child anchor=north}},
  red subtree/.style={for tree={text=red},for descendants={edge=red}},
  black subtree/.style={for tree={text=black},for descendants={edge=black}},
  blue subtree/.style={for tree={text=blue},for descendants={edge=blue}},
  green subtree/.style={for tree={text=green},for descendants={edge=green}},
  gray subtree/.style={for tree={text=gray},for descendants={edge=gray}},
  color subtree/.style={for tree={text=#1},for descendants={edge={draw=#1}}},
  subtree color/.style={for tree={text=#1},for descendants={edge={draw=#1}}},
  vcenter/.style={begin draw/.code={\begin{tikzpicture}[baseline=(current bounding box.center)]}},
  empty nodes/.style={	% from the forest manual
    for tree={
      % calign=fixed edge angles,
      yshift=1ex},
    delay={where content={}{shape=coordinate,for parent={for children={anchor=north}}}{}}},
  derivation tree/.style={.style={
      for tree={parent anchor={},child anchor={},font=\ttfamily}}},
  derivation/.style={for children={fit=tight}},
  dt label/.style 2 args={
    edge label={node[midway,font=\ttfamily\scriptsize, #1]{#2}},},
  %% for drawing STUG sequences
  %% conn, <-- and --> are take from https://tex.stackexchange.com/a/414444/61499
  conn/.style n args=2{
    tikz+={\draw [dashed] (#1) -- (#2);}},
  -->/.style={ % draw horizontal line to predecessor
    before drawing tree={
      temptoksa=,
      tempcounta/.option=level,
      if nodewalk valid={n}{conn={}{!n}}{
        for nodewalk={
          while nodewalk valid={u}{
            tempcountb/.option=n,
            u,
            if={> OR>{n children}{tempcountb} }{
              tempcountc/.register=tempcountb,
              tempcountc'+=1,
              n/.register=tempcountc,
              while nodewalk valid={1}{
                1,
                if={ > OR= {level}{tempcounta} }{
                  temptoksa/.option=name,
                  break=1
                }{}
              }
            }{}
          }
        }{},
        if temptoksa={}{}{conn/.process={_R {}{temptoksa}}}
      },
    },
  },
  <--/.style={ % draw horizontal line to successor
    before drawing tree={
      temptoksa=,
      tempcounta/.option=level,
      if nodewalk valid={p}{conn={}{!p}}{
        for nodewalk={
          while nodewalk valid={u}{
            u,
            if n=1{}{
              p,
              while nodewalk valid={1}{
                1,
                if={ > OR= {level}{tempcounta} }{
                  temptoksa/.option=name,
                  break=1
                }{}
              }
            },
          }
        }{},
        if temptoksa={}{}{conn/.process={_R {}{temptoksa}}}
      },
    },
  },
  t/.style={calign with current}, % trunk 
  lex/.style={                    % terminal nodes with lexical material
    no edge,
    for parent={l sep=0ex},
    yshift=3ex,
    draw=gray,
    content={\textit{##1}}},
  c/.style args={#1}{            % visual link with argument
    % edge label={node[xshift={0.8em},scale=0.8,fill=white,draw,inner sep=.10ex,circle]{#1}}
    % tikz={\node[yshift={1.5ex},scale=0.8,fill=white,draw,inner sep=.10ex,circle, right=-0.5em of .east]  {1};}
    label={[yshift={0.5ex},scale=.8,circle, draw, fill=white, inner sep=.1ex, label distance=-.65em, anchor=west]north east:#1}},
  Xs/.style={for tree={xshift=#1}},
  Ys/.style={for tree={yshift=#1}},
  %% Beamer overlays (https://tex.stackexchange.com/a/253384/61499)
  invisible/.style={%
    for tree={%
      /tikz/invisible={#1},
      edge={/tikz/invisible={#1}}}},
  visible/.style={%
    for tree={%
      /tikz/visible={#1},
      edge={/tikz/visible={#1}}}},
  opaque/.style={%
    for tree={%
      /tikz/opaque={#1},
      edge={/tikz/opaque={#1}}}},
  alert/.style={%
    for tree={%
      /tikz/alert={#1},
      edge={/tikz/alert={#1}}}},
  only/.code args={<#1>}{% https://tex.stackexchange.com/a/417110/61499
    \alt<#1>{}{\pgfkeysalso{before typesetting nodes={remove}}}},
}


\usepackage{url}
\usepackage{amsmath,amssymb,amsfonts,marvosym}
\usepackage{ulem}			% to cross out text
\normalem

\usepackage{ragged2e}
\let\raggedright=\RaggedRight

% \usepackage{tipa}
\usepackage{linguex}   % must be loaded below \usepackage[T1]{fontenc}
\AtBeginDocument{
  \setlength{\Exlabelsep}{0em}		% for linguex examples
  \setlength{\SubExleftmargin}{1,5em}	% for linguex examples
  \renewcommand\eachwordone{\sffamily}	% for glossing with linguex
  \renewcommand\eachwordtwo{\sffamily}	% for glossing with linguex
  % \setlength{\Extopsep}{1ex}   % vertical margin in linguex examples
}

%%%%%%%%%%%%%%%%%%%%%%
%   AVM SETTINGS     % 
%%%%%%%%%%%%%%%%%%%%%%

\usepackage{packages/avm}

\avmoptions{center} 
\avmfont{\scshape}
\avmvalfont{\normalfont}
\avmsortfont{\normalfont\itshape}

\newenvironment{topbot}{   	% more flexible than /newcommand ?
	\avmvskip{0.2ex} 
	\hspace{-1.5em}
	\begin{avm}
	\avml
	}
	%%%
	{
	\avmr
    \end{avm}
    \hspace{-0.5em}
}

%%%%%%%%%%%%%%%%%%%%%%%%
%   BEAMER STYLE    % 
%%%%%%%%%%%%%%%%%%%%%%%%

%\usefonttheme{serif}
%\renewcommand*{\ttdefault}{cmtt}

\definecolor{HHUblue}{HTML}{006AB3}
\setbeamercolor{structure}{fg=HHUblue}

\setbeamerfont{frametitle}{family=\sffamily}
\setbeamerfont{title}{family=\sffamily}
\setbeamerfont{block title}{family=\sffamily}

\usetheme{Copenhagen} % Boadilla
\usecolortheme{default}   % beaver
\usefonttheme{default}		% default | professionalfonts | serif | structurebold | structureitalicserif | structuresmallcapsserif
\useinnertheme{default} 	% circles | default | inmargin | rectangles | rounded
\useoutertheme{default}	% default | infolines | miniframes | shadow | sidebar | smoothbars | smoothtree | split | tree

%\setbeamercovered{transparent}				% for transparent overlays
\setbeamercovered{invisible}				% for non-transparent overlays
\setbeamertemplate{navigation symbols}{}	% no navigation symbols
\setbeamertemplate{headline}[default]		% no headline
\setbeamertemplate{footline}[frame number]
\setbeamertemplate{section in toc}[]
\setbeamertemplate{subsection in toc}[]
\setbeamertemplate{itemize items}[square]
\setbeamertemplate{enumerate items}[square]
%\setbeamertemplate{blocks}[default]		% rectangular blocks
%\setbeamersize{text margin left=10pt,text margin right=10pt}

%% Bibliography style (http://tex.stackexchange.com/questions/97615/article-style-bibliography-in-beamer-class)
\setbeamertemplate{frametitle continuation}[from second]
% Now get rid of all the colours
\setbeamercolor*{bibliography entry title}{fg=black}
\setbeamercolor*{bibliography entry author}{fg=black}
\setbeamercolor*{bibliography entry location}{fg=black}
\setbeamercolor*{bibliography entry note}{fg=black}
% and kill the abominable icon
\setbeamertemplate{bibliography item}{\insertbiblabel}  % insert label from bib(la)tex
\AtBeginDocument{
  \renewcommand*{\bibfont}{\scriptsize}
}

\tikzset{% makes available \only and \alt inside paths
  only/.code args={<#1>#2}{\only<#1>{\pgfkeysalso{#2}}},
  alt/.code args={<#1>#2#3}{\alt<#1>{\pgfkeysalso{#2}}{\pgfkeysalso{#3}}}
}

\setbeamertemplate{footline}
{
  \leavevmode%
  \hbox{%
    \pgfsetfillopacity{0}\begin{beamercolorbox}[wd=.333333\paperwidth,ht=2.25ex,dp=1ex,left]{author in head/foot}%
      \usebeamerfont{author in head/foot}\pgfsetfillopacity{1}\color{gray}\hspace*{2ex}\insertshortauthor~~(\insertshortinstitute)
    \end{beamercolorbox}%
    \pgfsetfillopacity{0}\begin{beamercolorbox}[wd=.333333\paperwidth,ht=2.25ex,dp=1ex,center]{title in head/foot}%
      \usebeamerfont{title in head/foot}\pgfsetfillopacity{1}\insertshorttitle
    \end{beamercolorbox}%
    \pgfsetfillopacity{0}\begin{beamercolorbox}[wd=.333333\paperwidth,ht=2.25ex,dp=1ex,right]{date in head/foot}%
    \usebeamerfont{date in head/foot}\pgfsetfillopacity{1}\color{gray}\insertshortdate{}\hspace*{2em}
      \insertframenumber{} %/ \inserttotalframenumber
      \hspace*{2ex}
    \end{beamercolorbox}}%
  \vskip0pt%
}


\newcommand{\separationframe}[1]{
\begin{frame}
\frametitle{}

\begin{center}
  \LARGE 
  \settowidth{\stmueTmp}{ #1 }
    \begin{minipage}{\stmueTmp}
    \begin{block}{}
    \begin{center}
    %\usebeamercolor[fg]{frametitle}
    #1
    \end{center}
    \end{block}
    \end{minipage}
\end{center}

\end{frame}
}

\newcommand\framecite[1]{
\vskip-2ex
\hfill #1%
\vskip-3.3ex ~
}

%% Bibliography

%% BibLaTeX
%%%%%%%%%%%%%%%%%%%%%%%%
%   BIBLATEX SETTINGS  % 
%%%%%%%%%%%%%%%%%%%%%%%%
\newcommand{\mycitestyle}{bst/biblatex-sp-unified/cbx/sp-authoryear-comp}
\makeatletter
\@ifclassloaded{beamer}{\renewcommand{\mycitestyle}{numeric-comp}}{}
\@ifclassloaded{tikzposter}{\renewcommand{\mycitestyle}{numeric-comp}}{}
\makeatother

\usepackage[
  natbib=true,
  style=bst/biblatex-sp-unified/bbx/biblatex-sp-unified,
  citestyle=\mycitestyle,
  %refsection=chapter,
  maxbibnames=99,
  isbn=false,
  doi=false,
  eprint=false,
  backend=biber,
  % sorting=ydnt,  % sort in descending chronological order
  indexing=cite,
  labelnumber,  % for numeric bibliography in beamer
  %toc=bib    % make bibliography appear in toc, incompatible with beamer
  ]{biblatex}
\renewcommand{\postnotedelim}{: }%
\renewcommand{\multicitedelim}{\addsemicolon\space}%
\renewcommand{\compcitedelim}{\multicitedelim}%
\DeclareFieldFormat{postnote}{#1}%

%% beamer settings
\makeatletter
\@ifclassloaded{beamer}{  
  \DeclareFieldFormat{labelnumberwidth}{[#1]}
  \defbibenvironment{bibliography}  % from numeric.bbx
      {\list
        {\printtext[labelnumberwidth]{%
          \printfield{prefixnumber}%
          \printfield{labelnumber}}}
        {\setlength{\labelwidth}{\labelnumberwidth}%
            \setlength{\leftmargin}{\labelwidth}%
            \setlength{\labelsep}{1em}%
            \addtolength{\leftmargin}{1em}%
            \setlength{\itemsep}{\bibitemsep}%
            \setlength{\parsep}{\bibparsep}}%
            \renewcommand*{\makelabel}[1]{\hss##1}}
      {\endlist}
      {\item}
    % \DeclareCiteCommand{\supercite}[\mkbibsuperscript]{
    %   \iffieldundef{prenote}
    %     {}
  %     {\BibliographyWarning{Ignoring prenote argument}}%
  %   \iffieldundef{postnote}
  %     {}
  %     {\BibliographyWarning{Ignoring postnote argument}}}
    %   {\usebibmacro{citeindex}%
  %      \color{gray}\bibopenbracket\usebibmacro{cite}\bibclosebracket}
    %   {\supercitedelim}
    %   {}
    \DeclareCiteCommand{\supercite}[\mkbibsuperscript]
      {\color{gray} % added color
      \usebibmacro{cite:init}%
      \let\multicitedelim=\supercitedelim
      \iffieldundef{prenote}
        {}
        {\BibliographyWarning{Ignoring prenote argument}}%
      \iffieldundef{postnote}
        {}
        {\BibliographyWarning{Ignoring postnote argument}}%
      \bibopenbracket}%
      {\usebibmacro{citeindex}%
       \usebibmacro{cite:comp}}
      {}
      {\usebibmacro{cite:dump}\bibclosebracket}

  \DeclareCiteCommand{\citeauthor}  % from sp-authoryear-comp.cbx; to add hyperref link  
    {\boolfalse{citetracker}%
     \boolfalse{pagetracker}%
     \usebibmacro{prenote}}
    {\ifciteindex
       {\indexnames{labelname}}
       {}%
     \printtext[bibhyperref]{\printnames{labelname}}}
    {\multicitedelim}
    {\usebibmacro{postnote}}

  \DeclareCiteCommand{\citeyear}  % from sp-authoryear-comp.cbx; to add hyperref link  
    {\boolfalse{citetracker}%
     \boolfalse{pagetracker}%
     \usebibmacro{prenote}}
    {\printfield[bibhyperref]{year}}
    {\multicitedelim}
    {\usebibmacro{postnote}}
}{}
\makeatother

%% tikzposter settings
\makeatletter
\@ifclassloaded{tikzposter}{  
\DeclareFieldFormat{labelnumberwidth}{#1}
\defbibenvironment{bibliography}
{\footnotesize\noindent}
{\unspace}
{}
\renewbibmacro*{begentry}{%
\textbf{\color{HHUblue}%
\printtext[labelnumberwidth]{%
[\printfield{prefixnumber}%
\printfield{labelnumber}]}%
\space
%\setunit{\addspace}
}}
\renewcommand*{\finentrypunct}{\addperiod\space}

\DeclareCiteCommand{\supercite}[\mkbibsuperscript]
{\color{gray} % added color
\usebibmacro{cite:init}%
\let\multicitedelim=\supercitedelim
\iffieldundef{prenote}
{}
{\BibliographyWarning{Ignoring prenote argument}}%
\iffieldundef{postnote}
{}
{\BibliographyWarning{Ignoring postnote argument}}%
\bibopenbracket}%
{\usebibmacro{citeindex}%
\usebibmacro{cite:comp}}
{}
{\usebibmacro{cite:dump}\bibclosebracket}

\DeclareCiteCommand{\citeauthor}  % from sp-authoryear-comp.cbx; to add hyperref link  
{\boolfalse{citetracker}%
\boolfalse{pagetracker}%
\usebibmacro{prenote}}
{\ifciteindex
{\indexnames{labelname}}
{}%
\printtext[bibhyperref]{\printnames{labelname}}}
{\multicitedelim}
{\usebibmacro{postnote}}

\DeclareCiteCommand{\citeyear}  % from sp-authoryear-comp.cbx; to add hyperref link  
{\boolfalse{citetracker}%
\boolfalse{pagetracker}%
\usebibmacro{prenote}}
{\printfield[bibhyperref]{year}}
{\multicitedelim}
{\usebibmacro{postnote}}
}{}
\makeatother

\addbibresource[datatype=bibtex]{references.bib}

\newcommand{\insertBib}{
  \printbibliography[
    %notkeyword=this
    ] 
}

\let\cite=\citet  % in order to prevent inconsistencies between \cite and \citet
\newcommand{\citeauthoryear}[1]{\citeauthor{#1} (\citeyear{#1})}
\newcommand{\citealtauthoryear}[1]{\citeauthor{#1} \citeyear{#1}}


%% BibTeX 
% \usepackage{natbib}
\setlength{\bibsep}{0mm}
%\setcitestyle{notesep={: }} 
\bibpunct[: ]{(}{)}{;}{a}{}{;}
\bibliographystyle{bst/unified}

\newcommand{\insertBib}{
	\bibliography{references}
}

\let\cite=\citet 	% in order to prevent inconsistencies between \cite and \citet 
% \PassOptionsToPackage{round}{natbib}
% \renewcommand{\newblock}{}    % to make natbib compatible with beamer

%%%%%%%%%%%%%%%%%%%%%%%%%
%   TEXTPOS SETTINGS    % 
%%%%%%%%%%%%%%%%%%%%%%%%%

\usepackage{calc}
\usepackage[absolute,overlay]{textpos}
\setlength{\TPHorizModule}{1em}
\setlength{\TPVertModule}{1ex}

\newlength{\textposOriginH}
\newlength{\textposOriginV}
\setlength{\textposOriginH}{10mm}
\setlength{\textposOriginV}{10mm}
\textblockorigin{\textposOriginH}{\textposOriginV}

\newcommand{\freeblock}[2]{
  \begin{textblock}{0.1}(#1)
  #2
  \end{textblock}
}
