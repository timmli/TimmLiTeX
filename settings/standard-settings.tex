%%%%%%%%%%%%%%%%%%%%%%%%
%   PDFLATEX SETTINGS  % 
%%%%%%%%%%%%%%%%%%%%%%%%

%% Font & Encoding
\usepackage{libertine}
\usepackage[libertine]{newtxmath}
\usepackage[scaled=0.8]{beramono}  % for monospaced font
\usepackage{microtype}		% micro-typographic aspects of the fonts
\usepackage[T1]{fontenc}	% special fonts, e.g. for German umlaute
%% incompabtible with Biblatex
% \usepackage{ucs}
% \usepackage[utf8x]{inputenc}
%% compatible with Biblatex
\usepackage[utf8]{inputenc}

%% Language
%\usepackage[german]{babel}
%\usepackage{german}
\usepackage[english]{babel}

%% Trees and graphics
\usepackage{graphics}
%%%%%%%%%%%%%%%%%%%%%%
%   TIKZ SETTINGS    % 
%%%%%%%%%%%%%%%%%%%%%%

\usepackage{tikz}
\usepackage{tikz-dependency}

\tikzset{every tree node/.style={align=center,anchor=north}}	% to allow linebreaks
\usetikzlibrary{calc} % for positioning arrows with ($(t.center)-(1,0)$)
\usetikzlibrary{shapes,decorations}
\usetikzlibrary{backgrounds,fit}
\usetikzlibrary{arrows}
\usetikzlibrary{matrix}
\usetikzlibrary{positioning}
\usetikzlibrary{automata}
\usetikzlibrary{tikzmark}

% Define box and box title style (see http://www.texample.net/tikz/examples/boxes-with-text-and-math/)
\tikzstyle{mybox} = [draw=gray, very thick,
    rectangle, rounded corners, inner sep=10pt, inner ysep=17pt,yshift=3pt]
\tikzstyle{fancytitle} =[draw=gray, very thick, fill=white,
    rectangle, rounded corners, inner sep=5pt, inner ysep=5pt]
\tikzstyle{mydouble} = [double distance=1pt]
    
\tikzset{
    %Define standard arrow tip
    >=stealth',
    %Define style for boxes
    box/.style={
           rectangle,
           rounded corners,
           draw=black, very thick,
           text width=10em,
           minimum height=2em,
           text centered},
    % Define arrow style
    arrow/.style={
           ->,
           thick,
           	shorten <=2pt,
           shorten >=2pt,}
}

\newcommand\centertikz[1]{\tikz[baseline=(current bounding box.center)]{#1}}
\newcommand\tikzcenter{baseline=(current bounding box.center)}
\newcommand\tikztop{baseline=(current bounding box.north)}

\newcommand\tikztreeset[1]{\matrix [matrix of nodes,left delimiter=\{,right delimiter=\}](set){#1};}
%%%%%%%%%%%%%%%%%%%%%%%
%   FOREST SETTINGS   % 
%%%%%%%%%%%%%%%%%%%%%%%
\usepackage{forest}

\makeatletter

\@ifpackagelater{forest}{2016/01/01}
{\useforestlibrary{linguistics}}
{}

\@ifpackagelater{forest}{2016/01/01}
{\newcommand{\forestPreamble}{default preamble}} % version >=2 of forest
{\newcommand{\forestPreamble}{.style}} % version <=1 of forest

\makeatother

\forestset{
  \forestPreamble ={
    % .style={ % version <=1 of forest
    % default preamble={ % version >=2 of forest    
		for tree={
			parent anchor=south, 
			child anchor=north,
			% align=center,			% bad: adds space below label
			fit=rectangle,
			base=top,				% vertical orientation of nodes
			% inner sep=3,			% necesssary?
			begin draw/.code={\begin{tikzpicture}[baseline=(current bounding box.center)]},
    }},
  sn edges/.style={for tree={parent anchor=south, child anchor=north}},
  red subtree/.style={for tree={text=red},for descendants={edge=red}},
  black subtree/.style={for tree={text=black},for descendants={edge=black}},
  blue subtree/.style={for tree={text=blue},for descendants={edge=blue}},
  green subtree/.style={for tree={text=green},for descendants={edge=green}},
  gray subtree/.style={for tree={text=gray},for descendants={edge=gray}},
  vcenter/.style={begin draw/.code={\begin{tikzpicture}[baseline=(current bounding box.center)]}},
  empty nodes/.style={	% from the forest manual
    for tree={
      % calign=fixed edge angles,
      yshift=1ex},
    delay={where content={}{shape=coordinate,for parent={for children={anchor=north}}}{}}},
  derivation tree/.style={.style={
      for tree={parent anchor={},child anchor={},font=\ttfamily}}},
  dt label/.style 2 args={
    edge label={node[midway,font=\ttfamily\scriptsize, #1]{#2}},},
  %% for drawing STUG sequences
  </.style={ % draw horizontal line to predecessor
    no edge,
    before drawing tree={tikz+={\draw[dashed](!)--(!p);}}
  },
  >/.style={ % draw horizontal line to successor
    no edge,
    before drawing tree={tikz+={\draw[dashed](!)--(!n);}}
  },
  t/.style={calign with current}, % trunk 
  lex/.style={                    % terminal nodes with lexical material
    no edge,
    for parent={l sep=0ex},
    yshift=3ex,
    draw=gray,
    content={\textit{##1}}},
  c/.style n args=1{            % visual link with argument
    % edge label={node[xshift={0.8em},scale=0.8,fill=white,draw,inner sep=.10ex,circle]{#1}}
    % tikz={\node[yshift={1.5ex},scale=0.8,fill=white,draw,inner sep=.10ex,circle, right=-0.5em of .east]  {1};}
    label={[yshift={0.5ex},scale=.8,circle, draw, fill=white, inner sep=.1ex, label distance=-.65em, anchor=west]north east:#1}
  }  
}

% \usepackage{arydshln} 		% for dashed horizontal lines in tables (incompatible with avm)
\usepackage{multirow}		% similar to \multicolumn

%% Symbols
\usepackage{latexsym,amsmath,amssymb,wasysym}
\usepackage{marvosym}		% for thunderbolt symbol
\usepackage{ulem}			% to cross out text
\normalem
\usepackage{url}
\urlstyle{sf}

% Language
% \usepackage[ngerman]{babel}	% uncomment for German

%% Linguistics
\usepackage{linguex}
\usepackage{tipa}	% for phonetic symbols
\renewcommand{\firstrefdash}{}
%%%%%%%%%%%%%%%%%%%%%%
%   AVM SETTINGS     % 
%%%%%%%%%%%%%%%%%%%%%%

\usepackage{packages/avm}

\avmoptions{center} 
\avmfont{\scshape}
\avmvalfont{\normalfont}
\avmsortfont{\normalfont\itshape}

\newenvironment{topbot}{   	% more flexible than /newcommand ?
	\avmvskip{0.2ex} 
	\hspace{-1.5em}
	\begin{avm}
	\avml
	}
	%%%
	{
	\avmr
    \end{avm}
    \hspace{-0.5em}
}
% \usepackage[inference]{semantic} % for CCG 
% \usepackage{packages/ccg}

%% Bibliography
%%%%%%%%%%%%%%%%%%%%%%%%
%   BIBLATEX SETTINGS  % 
%%%%%%%%%%%%%%%%%%%%%%%%
\newcommand{\mycitestyle}{bst/biblatex-sp-unified/cbx/sp-authoryear-comp}
\makeatletter
\@ifclassloaded{beamer}{\renewcommand{\mycitestyle}{numeric-comp}}{}
\@ifclassloaded{tikzposter}{\renewcommand{\mycitestyle}{numeric-comp}}{}
\makeatother

\usepackage[
  natbib=true,
  style=bst/biblatex-sp-unified/bbx/biblatex-sp-unified,
  citestyle=\mycitestyle,
  %refsection=chapter,
  maxbibnames=99,
  isbn=false,
  doi=false,
  eprint=false,
  %backend=biber,
  backend=biber,
  % sorting=ydnt,  % sort in descending chronological order
  indexing=cite,
  labelnumber,  % for numeric bibliography in beamer
  %toc=bib    % make bibliography appear in toc, incompatible with beamer
  ]{biblatex}
\renewcommand{\postnotedelim}{: }%
\renewcommand{\multicitedelim}{\addsemicolon\space}%
\renewcommand{\compcitedelim}{\multicitedelim}%
\DeclareFieldFormat{postnote}{#1}%

%% beamer settings
\makeatletter
\@ifclassloaded{beamer}{  
  \DeclareFieldFormat{labelnumberwidth}{[#1]}
  \defbibenvironment{bibliography}  % from numeric.bbx
      {\list
        {\printtext[labelnumberwidth]{%
          \printfield{prefixnumber}%
          \printfield{labelnumber}}}
        {\setlength{\labelwidth}{\labelnumberwidth}%
            \setlength{\leftmargin}{\labelwidth}%
            \setlength{\labelsep}{1em}%
            \addtolength{\leftmargin}{1em}%
            \setlength{\itemsep}{\bibitemsep}%
            \setlength{\parsep}{\bibparsep}}%
            \renewcommand*{\makelabel}[1]{\hss##1}}
      {\endlist}
      {\item}
    % \DeclareCiteCommand{\supercite}[\mkbibsuperscript]{
    %   \iffieldundef{prenote}
    %     {}
  %     {\BibliographyWarning{Ignoring prenote argument}}%
  %   \iffieldundef{postnote}
  %     {}
  %     {\BibliographyWarning{Ignoring postnote argument}}}
    %   {\usebibmacro{citeindex}%
  %      \color{gray}\bibopenbracket\usebibmacro{cite}\bibclosebracket}
    %   {\supercitedelim}
    %   {}
    \DeclareCiteCommand{\supercite}[\mkbibsuperscript]
      {\color{gray} % added color
      \usebibmacro{cite:init}%
      \let\multicitedelim=\supercitedelim
      \iffieldundef{prenote}
        {}
        {\BibliographyWarning{Ignoring prenote argument}}%
      \iffieldundef{postnote}
        {}
        {\BibliographyWarning{Ignoring postnote argument}}%
      \bibopenbracket}%
      {\usebibmacro{citeindex}%
       \usebibmacro{cite:comp}}
      {}
      {\usebibmacro{cite:dump}\bibclosebracket}

  \DeclareCiteCommand{\citeauthor}  % from sp-authoryear-comp.cbx; to add hyperref link  
    {\boolfalse{citetracker}%
     \boolfalse{pagetracker}%
     \usebibmacro{prenote}}
    {\ifciteindex
       {\indexnames{labelname}}
       {}%
     \printtext[bibhyperref]{\printnames{labelname}}}
    {\multicitedelim}
    {\usebibmacro{postnote}}

  \DeclareCiteCommand{\citeyear}  % from sp-authoryear-comp.cbx; to add hyperref link  
    {\boolfalse{citetracker}%
     \boolfalse{pagetracker}%
     \usebibmacro{prenote}}
    {\printfield[bibhyperref]{year}}
    {\multicitedelim}
    {\usebibmacro{postnote}}
}{}
\makeatother

%% tikzposter settings
\makeatletter
\@ifclassloaded{tikzposter}{  
\DeclareFieldFormat{labelnumberwidth}{#1}
\defbibenvironment{bibliography}
{\footnotesize\noindent}
{\unspace}
{}
\renewbibmacro*{begentry}{%
\textbf{\color{HHUblue}%
\printtext[labelnumberwidth]{%
[\printfield{prefixnumber}%
\printfield{labelnumber}]}%
\space
%\setunit{\addspace}
}}
\renewcommand*{\finentrypunct}{\addperiod\space}

\DeclareCiteCommand{\supercite}[\mkbibsuperscript]
{\color{gray} % added color
\usebibmacro{cite:init}%
\let\multicitedelim=\supercitedelim
\iffieldundef{prenote}
{}
{\BibliographyWarning{Ignoring prenote argument}}%
\iffieldundef{postnote}
{}
{\BibliographyWarning{Ignoring postnote argument}}%
\bibopenbracket}%
{\usebibmacro{citeindex}%
\usebibmacro{cite:comp}}
{}
{\usebibmacro{cite:dump}\bibclosebracket}

\DeclareCiteCommand{\citeauthor}  % from sp-authoryear-comp.cbx; to add hyperref link  
{\boolfalse{citetracker}%
\boolfalse{pagetracker}%
\usebibmacro{prenote}}
{\ifciteindex
{\indexnames{labelname}}
{}%
\printtext[bibhyperref]{\printnames{labelname}}}
{\multicitedelim}
{\usebibmacro{postnote}}

\DeclareCiteCommand{\citeyear}  % from sp-authoryear-comp.cbx; to add hyperref link  
{\boolfalse{citetracker}%
\boolfalse{pagetracker}%
\usebibmacro{prenote}}
{\printfield[bibhyperref]{year}}
{\multicitedelim}
{\usebibmacro{postnote}}
}{}
\makeatother

\addbibresource[datatype=bibtex]{references.bib}

\newcommand{\insertBib}{
  \printbibliography[
    %notkeyword=this
    ] 
}

\let\cite=\citet  % in order to prevent inconsistencies between \cite and \citet
\newcommand{\citeauthoryear}[1]{\citeauthor{#1} (\citeyear{#1})}
\newcommand{\citealtauthoryear}[1]{\citeauthor{#1} \citeyear{#1}}
 
% \usepackage{natbib}
\setlength{\bibsep}{0mm}
%\setcitestyle{notesep={: }} 
\bibpunct[: ]{(}{)}{;}{a}{}{;}
\bibliographystyle{bst/unified}

\newcommand{\insertBib}{
	\bibliography{references}
}

\let\cite=\citet 	% in order to prevent inconsistencies between \cite and \citet 

% Hyperrefs in PDF 
% Hyperrefs in PDF 
\usepackage[bookmarks=true,bookmarksopen=true,%
  hyperindex=true,%
  breaklinks=true,
  draft=false,plainpages=false,
  pdfauthor={},%
  pdfkeywords={},%
  ]{hyperref}
\hypersetup{colorlinks=false, pdfborder={0 0 0}}

%% for blindtext
\usepackage{lipsum}
