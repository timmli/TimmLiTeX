%%%%%%%%%%%%%%%%%%%%%%
%   TIKZ SETTINGS    % 
%%%%%%%%%%%%%%%%%%%%%%

\usepackage{tikz}

\tikzset{every tree node/.style={align=center,anchor=north}}	% to allow linebreaks
\usetikzlibrary{calc} % for positioning arrows with ($(t.center)-(1,0)$)
\usetikzlibrary{shapes,decorations}
\usetikzlibrary{backgrounds,fit}
\usetikzlibrary{arrows}
\usetikzlibrary{matrix}
\usetikzlibrary{positioning}
\usetikzlibrary{automata}
\usetikzlibrary{tikzmark}

% Define box and box title style (see http://www.texample.net/tikz/examples/boxes-with-text-and-math/)
\tikzstyle{mybox} = [draw=gray, very thick,
    rectangle, rounded corners, inner sep=10pt, inner ysep=17pt,yshift=3pt]
\tikzstyle{fancytitle} =[draw=gray, very thick, fill=white,
    rectangle, rounded corners, inner sep=5pt, inner ysep=5pt]
\tikzstyle{mydouble} = [double distance=1pt]
    
\tikzset{
    %Define standard arrow tip
    >=stealth',
    %Define style for boxes
    box/.style={
           rectangle,
           rounded corners,
           draw=black, very thick,
           text width=10em,
           minimum height=2em,
           text centered},
    % Define arrow style
    arrow/.style={
           ->,
           thick,
           	shorten <=2pt,
           shorten >=2pt,}
}

\newcommand\centertikz[1]{\tikz[baseline=(current bounding box.center)]{#1}}
\newcommand\tikzcenter{baseline=(current bounding box.center)}
\newcommand\tikztop{baseline=(current bounding box.north)}

\newcommand\tikztreeset[1]{\matrix [matrix of nodes,left delimiter=\{,right delimiter=\}](set){#1};}