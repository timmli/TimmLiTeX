\begin{frame}
  \frametitle{Overlays}
  {\small 

    Explicit specification of overlays:
    \begin{itemize}
    \item \color<2>{red}{Changing colors ...}
    \item \alert<3>{Alert mode ...} 
    \item \textbf<4>{Changing the font face ...}
    \item \only<-5>{Changing existence ...} 
    \item \visible<-6>{Changing visability ...}
    \item \uncover<7->{Uncovering from grey ...}
    \item \alt<8>{Specifying alternations ...}{... in one instruction}
    \end{itemize}

    \vfill
    {\tt $\backslash$pause[<number>]} separates two overlays.

    \vfill
    Overlays in list environments:
    \begin{itemize}
    \item<9-> First item
    \item<alert@10> Second item with alert
    \item Alternatively, list environments can have an overlay option such as {\tt <+-> } or {\tt <+- alert@ +>}.
    \end{itemize}

    \vfill
    Remember that you can declare overlays in the frame options.
    \hyperlink{frameoptions}{\beamerbutton{The frame environment}}

  }
\end{frame}


\begin{frame}
\frametitle{Overlays in Forest trees}

%% This example may cause problems together with TeX Live 2016
% \Forest{
%   [,phantom
%     [S,only=<+> [NP [N [I,lex]]]]
%     [S,only=<+>,--> [NP [N [I,lex]]]]
%     [S,only=<1-.> [V [wrote,lex]]]
%     [S,only=<+>
%       [NP [N [I,lex]]]
%       [V [wrote,lex]]]
%     [S,only=<+>,-->
%       [NP [N [I,lex]]]
%       [V [wrote,lex]]]
%     [S,only=<1-.> [NP [D [a,lex]]]]
%     [S,only=<+>
%       [NP [N [I,lex]]]
%       [V [wrote,lex]]
%       [NP [D [a,lex]]]]
%     [S,only=<+>,-->
%       [NP [N [I,lex]]]
%       [V [wrote,lex]]
%       [NP,--> [D [a,lex]]]]
%     [S,only=<1-.> [NP [N [novel,lex]]]]
%     [S,only=<+>
%       [NP [N [I,lex]]]
%       [V [wrote,lex]]
%       [NP
%         [D [a,lex]]
%         [N [novel,lex]]]]
%     [S,only=<+->,alert=<.>,c=0
%       [NP,c=1 [N [I,lex]]]
%       [V [wrote,lex]]
%       [NP,c=2
%         [D [a,lex]]
%         [N [novel,lex]]]] 
%     ]
%   }

\end{frame}
