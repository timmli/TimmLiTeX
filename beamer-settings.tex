%%%%%%%%%%%%%%%%%%%%%%%%
%   PDFLATEX SETTINGS  % 
%%%%%%%%%%%%%%%%%%%%%%%%

%%   Font & Encoding

% \usepackage{libertine} % libertine tends to cause problems, e.g. when using tipa
% \usepackage[libertine]{newtxmath}
\usepackage{times}

\makeatletter

\@ifclassloaded{beamer}{
  \usepackage{libertine} % libertine tends to cause problems, e.g. when using tipa
  \usepackage[libertine]{newtxmath}}{}

\@ifclassloaded{tikzposter}{
  \usepackage{DejaVuSans} 
  \renewcommand*{\familydefault}{\sfdefault}}{}

\makeatother

\usepackage[scaled=0.8]{beramono}  % for monospaced font
\usepackage{microtype}		% micro-typographic aspects of the fonts
\usepackage[T1]{fontenc}	% special fonts, e.g. for German umlaute

%% compatible with BibLaTeX
\usepackage[utf8]{inputenc}
%% incompabtible with BibLaTeX
% \usepackage{ucs}
% \usepackage[utf8x]{inputenc}

%% Language
%\usepackage{german} % is this useful anymore?
\usepackage[german,english]{babel} % the last language in the options is loaded; the other one can be chosen with \selectlanguage
\usepackage{iflang}                % for language specific settings

%%%%%%%%%%%%%%%%%%%%%%%%
%   MISCELLANEOUS      % 
%%%%%%%%%%%%%%%%%%%%%%%%

%% Graphics
\usepackage{graphics}

%% Tables
% \usepackage{arydshln} 		    % for dashed horizontal lines in tables (incompatible with avm)
\usepackage{multirow}           % similar to \multicolumn

%% Symbols
\usepackage{latexsym,amsmath,amssymb,wasysym}
\usepackage{marvosym}           % for thunderbolt symbol
\usepackage{ulem}               % to cross out text
  \normalem
\usepackage{url}
  \urlstyle{tt}                   % tt,rm,sf,same
\usepackage{tipa}	% for phonetic symbols; has to appear before fontspec

%% Blindtext
\usepackage{lipsum}

%% Set ragged text
\usepackage{ragged2e}
\let\raggedright=\RaggedRight

%%%%%%%%%%%%%%%%%%%%%%%% 
% BIBLATEX SETTINGS  % 
%%%%%%%%%%%%%%%%%%%%%%%% 

\newcommand{\mycitestyle}{muss}
\newcommand{\mybibstyle}{muss}

\makeatletter
\@ifclassloaded{beamer}{\renewcommand{\mycitestyle}{numeric-comp}}{}
\@ifclassloaded{tikzposter}{\renewcommand{\mycitestyle}{numeric-comp}}{}
\makeatother

\usepackage[
  bibstyle=\mybibstyle,
  citestyle=\mycitestyle,
  %% The remaining options are set in muss.bbx.
  % natbib=true,
  % refsection=chapter,
  % maxbibnames=99,
  % isbn=false,
  % doi=false,
  % eprint=false,
  % backend=biber,
  % sorting=ydnt,  % sort in descending chronological order
  % indexing=cite,
  % labelnumber,  % for numeric bibliography in beamer
  % toc=bib    % make bibliography appear in toc, incompatible with beamer
  ]{biblatex}

%% The bibliography file is specified like this
\addbibresource[datatype=bibtex]{references.bib}

%% Command for inserting bibliography here 
\newcommand{\insertBib}{
  \printbibliography[
    % notkeyword=this
    ] 
}

%% Compat definitions to make available BibTeX macros in BibLaTeX 
% \let\citealt=\cite
% \let\cite=\textcite
% \let\citep=\parencite
% \let\citet=\cite
% \newcommand{\citeauthoryear}[1]{\citeauthor{#1} (\citeyear{#1})}
% \newcommand{\citealtauthoryear}[1]{\citeauthor{#1} \citeyear{#1}}



% Beamer settings
%-----------------
\makeatletter
\@ifclassloaded{beamer}{

  \ExecuteBibliographyOptions{labelnumber}
  
  %% Print squared brackets around reference number
  %\DeclareFieldFormat{labelnumberwidth}{\mkbibbrackets{#1}}

  %% Print ": " before postnote (numeric-comp uses comma) 
  \renewcommand*{\postnotedelim}{\addcolon\space} 
  
  %% Taken from numeric.bbx
  \defbibenvironment{bibliography}  
  {\list
    {\printtext[labelnumberwidth]{%
        % \printfield{prefixnumber}%
        \printfield{labelnumber}}}
    {\setlength{\labelwidth}{\labelnumberwidth}%
      \setlength{\leftmargin}{\labelwidth}%
      \setlength{\labelsep}{1em}%
      \addtolength{\leftmargin}{1em}%
      \setlength{\itemsep}{\bibitemsep}%
      \setlength{\parsep}{\bibparsep}}%
    \renewcommand*{\makelabel}[1]{\hss##1}}
  {\endlist}
  {\item}

  %% Taken from MUSS v0.4
  \DeclareCiteCommand{\fullcite}
  {\defcounter{maxnames}{\blx@maxbibnames}% show all names
    \usebibmacro{prenote}}
  {\usedriver
    {\DeclareNameAlias{sortname}{default}}
    {\thefield{entrytype}}}
  {\multicitedelim}
  {\usebibmacro{postnote}}

  %% Taken from numberic-comp.cbx with additions ...
  \DeclareCiteCommand{\supercite}[\mkbibsuperscript]
  {\color{gray}% added color
    \usebibmacro{cite:init}%
    \let\multicitedelim=\supercitedelim
    \let\multicitesubentrydelim=\supercitesubentrydelim
    \let\multiciterangedelim=\superciterangedelim
    \let\multicitesubentryrangedelim=\supercitesubentryrangedelim
    \iffieldundef{prenote}
    {}
    {\BibliographyWarning{Ignoring prenote argument}}%
    \iffieldundef{postnote}
    {}
    {\BibliographyWarning{Ignoring postnote argument}}%
    \bibopenbracket% added bracket
  }
  {\usebibmacro{citeindex}%
    \usebibmacro{cite:comp}}
  {}
  {\usebibmacro{cite:dump}%
    \bibclosebracket% added bracket
  }
  
}{}
\makeatother\makeatother

%%%%%%%%%%%%%%%%%%%%%%%%%
%   TEXTPOS SETTINGS    % 
%%%%%%%%%%%%%%%%%%%%%%%%%

\usepackage{calc} % Enables the use of calc expressions: 0pt+1cm
\usepackage[
  absolute, % Absolute positioning
  overlay   % Positioned textblocks will overlay any other parts of the text.  
]{textpos}

\setlength{\TPHorizModule}{1em}
\setlength{\TPVertModule}{1ex}
\newlength{\textposOriginH}
\newlength{\textposOriginV}
\setlength{\textposOriginH}{10mm}
\setlength{\textposOriginV}{10mm}
\textblockorigin{\textposOriginH}{\textposOriginV}

\newcommand{\freeblock}[2]{
  \begin{textblock}{0.1}(#1)
  #2
  \end{textblock}
}
