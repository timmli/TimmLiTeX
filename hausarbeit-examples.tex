\section{Information about this template}

More \LaTeX{} templates on \url{https://github.com/timmli/latex-templates}.

\section{Examples}

\subsection{Citation}

The following is a alist of available cite commands. See \url{https://ctan.space-pro.be/tex-archive/macros/latex/contrib/biblatex/doc/biblatex.pdf#subsection.3.9} for a more detailed description.

\begin{itemize}
\item \verb=\nocite= \nocite{Bech:63}
\item \verb=\fullcite[<prenote>][<postnote>]{Nallapati:etal:16}= \newline  \fullcite[<prenote>][<postnote>]{Nallapati:etal:16}
\item \verb=\textcite[<prenote>][<postnote>]{Nallapati:etal:16}= \newline \textcite[<prenote>][<postnote>]{Nallapati:etal:16}
\item \verb=\parencite[<prenote>][<postnote>]{Nallapati:etal:16}= \newline \parencite[<prenote>][<postnote>]{Nallapati:etal:16}
\item \verb=\parencite*[<prenote>][<postnote>]{Nallapati:etal:16}= \newline \parencite*[<prenote>][<postnote>]{Nallapati:etal:16}
\item \verb=\smartcite[<prenote>][<postnote>]{Nallapati:etal:16}= \newline \smartcite[<prenote>][<postnote>]{Nallapati:etal:16}
\item \verb=\footcite[<prenote>][<postnote>]{Nallapati:etal:16}= \newline \footcite[<prenote>][<postnote>]{Nallapati:etal:16}
\item \verb=\footfullcite[<prenote>][<postnote>]{Nallapati:etal:16}= \newline  \footfullcite[<prenote>][<postnote>]{Nallapati:etal:16}
\item \verb=\supercite[<prenote>][<postnote>]{Nallapati:etal:16}= \newline \supercite[<prenote>][<postnote>]{Nallapati:etal:16}
\item \verb=\footcitetext[<prenote>][<postnote>]{Nallapati:etal:16}= \newline \footcitetext[<prenote>][<postnote>]{Nallapati:etal:16}
\item \verb=\autocite[<prenote>][<postnote>]{Nallapati:etal:16}= \newline  \autocite[<prenote>][<postnote>]{Nallapati:etal:16}
\item \verb=\cite[<prenote>][<postnote>]{Nallapati:etal:16}= \newline  \cite[<prenote>][<postnote>]{Nallapati:etal:16}
\item \verb=\cite*[<prenote>][<postnote>]{Nallapati:etal:16}= \newline  \cite*[<prenote>][<postnote>]{Nallapati:etal:16}
\item \verb=\citeauthor[<prenote>][<postnote>]{Nallapati:etal:16}= \newline \citeauthor[<prenote>][<postnote>]{Nallapati:etal:16}
\item \verb=\citeauthor*[<prenote>][<postnote>]{Nallapati:etal:16}= \newline \citeauthor*[<prenote>][<postnote>]{Nallapati:etal:16}
\item \verb=\citeurl[<prenote>][<postnote>]{Nallapati:etal:16}= \newline \citeurl[<prenote>][<postnote>]{Nallapati:etal:16}
\item \verb=\citetitle[<prenote>][<postnote>]{Nallapati:etal:16}= \newline \citetitle[<prenote>][<postnote>]{Nallapati:etal:16}
\item \verb=\citeyear[<prenote>][<postnote>]{Nallapati:etal:16}= \newline \citeyear[<prenote>][<postnote>]{Nallapati:etal:16}
\end{itemize} 

References are stored in \texttt{references.bib}.

\subsection{Code listings}

\begin{lstlisting}[language=Python,
  % float,
  % caption=A floating example with Python code
  ]
from numpy import random
import matplotlib.pyplot as plt
import seaborn as sns

sns.distplot(random.normal(scale=2, size=1000), hist=False, label='normal')
sns.distplot(random.logistic(size=1000), hist=False, label='logistic')

plt.show()
\end{lstlisting}

\subsection{AVM}

\avm{
  \0 [ \type{eating} \\
    actor & \1 \\ 
    theme & \2 ]
}

\medskip

\noindent $\Rightarrow$ \url{https://github.com/langsci/langsci-avm/blob/master/langsci-avm.pdf}

\subsection{Linguistic examples}

\ex. This is a simple example.

\exg. [Noch am Boden liegend$_i$], sei [auf ihn$_i$] eingetreten worden.\\
still on.the floor lying be on him PART.kicked got\\
`While he was still on the floor he was kicked.'\\
(Cf. (422) in \cite{Mueller:02})

\noindent $\Rightarrow$ \url{http://texdoc.net/texmf-dist/doc/latex/linguex/linguex-doc.pdf} \\
Note the Leipzig glossing rules: \url{http://www.eva.mpg.de/lingua/resources/glossing-rules.php}

\ex. \textipa{[""Ekspl@"neIS@n]}

$\Rightarrow$ \url{http://en.wikibooks.org/wiki/LaTeX/Linguistics#IPA_characters}

\subsection{Math formulae}

Formulae in texts: $a^2 + b^2 = c^2$

\noindent Formulae in equation environment:
\begin{equation}
a^2 + b^2 = c^2
\end{equation}
$\Rightarrow$ \url{http://en.wikibooks.org/wiki/LaTeX/Mathematics}

\subsection{Tables}

\begin{tabular}{c|c|c}
\hline
cell 11 & cell 12 & cell 13 \\
\hline
cell 21 & cell 22 & cell 23 \\
\hline
\end{tabular}

\subsection{Todonotes}

The following are examples of using the todonotes package. 

\begin{itemize}
\item The most relevant command is probably \verb|\todo[<options>]{<todo>}|:
\todo[
  % ,author=Xavier
  % ,color=green
  % ,inline
  % ,fancyline
  % ,size=\scriptsize
  ]{Do something!}example with \verb|\todo| 
\item Additionally, I have defined \verb|\todoregion{<region>}{<todo>}| for highlighting the region of the todo:
\todoregion{example}{Do something!} with \verb|\todoregion|
\end{itemize}

\noindent The list of todos can be printed with \verb|\listoftodos|.

\subsection{Trees}

\Forest{
  [,phantom
    [S,name=root,red subtree
      [AUX [\textit{does}] ]
      [S
        [NP]
        [VP,name=vp[V [\textit{say}]] [S*,name=foot]]]]
    [S,Xs=5cm,Ys=2cm
      [NP]
      [S,name=s
        [NP [$\varepsilon$] ]
        [VP 
          [VP [V [\textit{walks}]]] 
          [PP]]]]
    ]
  \draw[->,gray,thick,dashed,bend right=10](root)to(s);
  \draw[->,gray,thick,dashed,bend right=10](foot)to(s);
}

\noindent $\Rightarrow$ \url{http://mirrors.ctan.org/graphics/pgf/contrib/forest/forest-doc.pdf}
