\begin{frame}
  \frametitle{About the template}
  
More \LaTeX{} templates on \url{https://github.com/timmli/latex-templates}.
  
\end{frame}

\begin{frame}[fragile]
\frametitle{Inline highlighting}
This is an \bsp{inline example} included with \verb|\bsp|.
This is how \term{terminology} can be introduced with \verb|\term|.
\end{frame}

\begin{frame}[label=frameoptions]
\frametitle{The frame environment}

  \hspace{-1em}{\tt $\backslash$begin\{frame\}[OVERLAY OPTION][OPTIONS]\{Title\}\{Subtitle\}}

  \bigskip

  {\bf OVERLAY OPTION:}
  \begin{itemize}
  \item {\tt <+->} 
  \end{itemize}

  \bigskip

  {\bf OPTIONS:}
  \begin{itemize}
  \item {\tt b,c,t}: frame orientation
  \item {\tt squeeze}: minimizes vertical margins
  \item {\tt plain}: suppresses title, header and sidebar
  \item {\tt label=name}: makes a frame reusable with {\tt $\backslash$againframe\{name\}} and can be used with hyperlinks.
  \item {\tt allowframebreaks}: spread frame content over several slides 
  \end{itemize}

  \vfill 

  Hyperlink: \hyperlink{frameoptions}{\beamerbutton{The frame environment}}

\end{frame}
\begin{frame}
  \frametitle{Blocks}

  \begin{block}{Title of a block}		% Titles may be left blank!
    Body of a block
  \end{block}

  \begin{exampleblock}{Title of an exampleblock}
    Body of an exampleblock
  \end{exampleblock}

  \begin{alertblock}{Title of an alertblock}
    Body of an alertblock
  \end{alertblock}
  
\end{frame}
\begin{frame}[fragile]
  \frametitle{Overlays}
  {\small 

    Explicit specification of overlays:
    \begin{itemize}
    \item \color<2>{red}{Changing colors ...}
    \item \alert<3>{Alert mode ...} 
    \item \textbf<4>{Changing the font face ... (requires [fragile])}
    \item \only<-5>{Changing existence ...} 
    \item \visible<-6>{Changing visability ...}
    \item \uncover<7->{Uncovering from grey ...}
    \item \alt<8>{Specifying alternations ...}{... in one instruction}
    \end{itemize}

    \vfill
    {\tt $\backslash$pause[<number>]} separates two overlays.

    \vfill
    Overlays in list environments:
    \begin{itemize}
    \item<9-> First item
    \item<alert@10> Second item with alert
    \item Alternatively, list environments can have an overlay option such as {\tt <+-> } or {\tt <+- alert@ +>}.
    \end{itemize}

    \vfill
    Remember that you can declare overlays in the frame options.
    \hyperlink{frameoptions}{\beamerbutton{The frame environment}}

  }
\end{frame}
\begin{frame}
  \frametitle{Columns}

  \begin{columns}
    \column{.55\textwidth}
    First column
		\pgfimage[width=\textwidth]{graphics/HHU-logo}
    \column{.45\textwidth}
		Second column
    \begin{enumerate}
    \item bla
    \item blupp
    \end{enumerate}
  \end{columns}

\end{frame}
\begin{frame}[fragile]
  \frametitle{Citations}

% The following is a alist of available cite commands. See \url{https://ctan.space-pro.be/tex-archive/macros/latex/contrib/biblatex/doc/biblatex.pdf#subsection.3.9} for a more detailed description.

\begin{itemize}
% \item \verb|\nocite| \nocite{Bech:63}
\item \verb|\fullcite[<prenote>][<postnote>]{Nallapati:etal:16}| \newline  \fullcite[<prenote>][<postnote>]{Nallapati:etal:16}
\item \verb|\footshortcite[<prenote>][<postnote>]{Nallapati:etal:16}| \newline  \footshortcite[<prenote>][<postnote>]{Nallapati:etal:16}
\item \verb|\textcite[<prenote>][<postnote>]{Nallapati:etal:16}| \newline \textcite[<prenote>][<postnote>]{Nallapati:etal:16}
\item \verb|\parencite[<prenote>][<postnote>]{Nallapati:etal:16}| \newline \parencite[<prenote>][<postnote>]{Nallapati:etal:16}
% \item \verb|\parencite*[<prenote>][<postnote>]{Nallapati:etal:16}| \newline \parencite*[<prenote>][<postnote>]{Nallapati:etal:16}
% \item \verb|\smartcite[<prenote>][<postnote>]{Nallapati:etal:16}| \newline \smartcite[<prenote>][<postnote>]{Nallapati:etal:16}
% \item \verb|\footcite[<prenote>][<postnote>]{Nallapati:etal:16}| \newline \footcite[<prenote>][<postnote>]{Nallapati:etal:16}
% \item \verb|\footfullcite[<prenote>][<postnote>]{Nallapati:etal:16}| \newline  \footfullcite[<prenote>][<postnote>]{Nallapati:etal:16}
\item \verb|\supercite[<prenote>][<postnote>]{Nallapati:etal:16}| \newline \supercite[<prenote>][<postnote>]{Nallapati:etal:16}
% \item \verb|\footcitetext[<prenote>][<postnote>]{Nallapati:etal:16}| \newline \footcitetext[<prenote>][<postnote>]{Nallapati:etal:16}
% \item \verb|\autocite[<prenote>][<postnote>]{Nallapati:etal:16}| \newline  \autocite[<prenote>][<postnote>]{Nallapati:etal:16}
% \item \verb|\cite[<prenote>][<postnote>]{Nallapati:etal:16}| \newline  \cite[<prenote>][<postnote>]{Nallapati:etal:16}
% \item \verb|\cite*[<prenote>][<postnote>]{Nallapati:etal:16}| \newline  \cite*[<prenote>][<postnote>]{Nallapati:etal:16}
% \item \verb|\citeauthor[<prenote>][<postnote>]{Nallapati:etal:16}| \newline \citeauthor[<prenote>][<postnote>]{Nallapati:etal:16}
% \item \verb|\citeauthor*[<prenote>][<postnote>]{Nallapati:etal:16}| \newline \citeauthor*[<prenote>][<postnote>]{Nallapati:etal:16}
% \item \verb|\citeurl[<prenote>][<postnote>]{Nallapati:etal:16}| \newline \citeurl[<prenote>][<postnote>]{Nallapati:etal:16}
% \item \verb|\citetitle[<prenote>][<postnote>]{Nallapati:etal:16}| \newline \citetitle[<prenote>][<postnote>]{Nallapati:etal:16}
% \item \verb|\citeyear[<prenote>][<postnote>]{Nallapati:etal:16}| \newline \citeyear[<prenote>][<postnote>]{Nallapati:etal:16}
\end{itemize} 

\end{frame}

\begin{frame}[fragile]
\frametitle{Linguistic examples}
\ex. This is a simple example.

\exg. [Noch am Boden liegend$_i$], sei [auf ihn$_i$] eingetreten worden.\\
still on.the floor lying be on him PART.kicked got\\
`While he was still on the floor he was kicked.'\\
(Cf. (422) in \cite{Mueller:02})

\noindent $\Rightarrow$ \url{http://texdoc.net/texmf-dist/doc/latex/linguex/linguex-doc.pdf} \\
Note the Leipzig glossing rules: \url{http://www.eva.mpg.de/lingua/resources/glossing-rules.php}

\ex. \textipa{[""Ekspl@"neIS@n]}

$\Rightarrow$ \url{http://en.wikibooks.org/wiki/LaTeX/Linguistics#IPA_characters}
\end{frame}

\begin{frame}
\frametitle{Trees}
% \Forest{
%   [S [NP] 
%     [VP [V  [\textit{eats}] ]
%       [NP] ]]
% }

%% This example may cause problems together with TeX Live 2016
% \Forest{
%   [,phantom
%     [S,--> [NP,--> [D [the]]]]
%     [S,--> [NP,--> [A [absent]]]]
%     [S, [NP [N [student]]]]
%     ]
% } 

\Forest{
  [,phantom
    [S,name=root,red subtree
      [AUX [\textit{does}] ]
      [S
        [NP]
        [VP,name=vp[V [\textit{say}]] [S*,name=foot]]]]
    [S,Xs=5cm,Ys=2cm
      [NP]
      [S,name=s
        [NP [$\varepsilon$] ]
        [VP 
          [VP [V [\textit{walks}]]] 
          [PP]]]]
    ]
  \draw[->,gray,thick,dashed,bend right=10](root)to(s);
  \draw[->,gray,thick,dashed,bend right=10](foot)to(s);
}

\noindent $\Rightarrow$ \url{http://mirrors.ctan.org/graphics/pgf/contrib/forest/forest-doc.pdf}
\end{frame}

\begin{frame}
\frametitle{Overlays in Forest trees}

%% This example may cause problems together with TeX Live 2016
A TUCO derivation:\newline

\Forest{
  [,phantom
    [S,only=<+> [NP [N [I,lex]]]]
    [S,only=<+>,--> [NP [N [I,lex]]]]
    [S,only=<1-.> [V [wrote,lex]]]
    [S,only=<+>
      [NP [N [I,lex]]]
      [V [wrote,lex]]]
    [S,only=<+>,-->
      [NP [N [I,lex]]]
      [V [wrote,lex]]]
    [S,only=<1-.> [NP [D [a,lex]]]]
    [S,only=<+>
      [NP [N [I,lex]]]
      [V [wrote,lex]]
      [NP [D [a,lex]]]]
    [S,only=<+>,-->
      [NP [N [I,lex]]]
      [V [wrote,lex]]
      [NP,--> [D [a,lex]]]]
    [S,only=<1-.> [NP [N [novel,lex]]]]
    [S,only=<+>
      [NP [N [I,lex]]]
      [V [wrote,lex]]
      [NP
        [D [a,lex]]
        [N [novel,lex]]]]
    [S,only=<+->,alert=<.>,c=0
      [NP,c=1 [N [I,lex]]]
      [V [wrote,lex]]
      [NP,c=2
        [D [a,lex]]
        [N [novel,lex]]]] 
    ]
  }
\end{frame}

\begin{frame}
\frametitle{AVM}
\avm{
  \0 [ \type{eating} \\
    actor & \1 \\ 
    theme & \2 ]
}

\medskip

\noindent $\Rightarrow$ \url{https://github.com/langsci/langsci-avm/blob/master/langsci-avm.pdf}
\end{frame}

% \begin{frame}
% \frametitle{IPA symbols}
% \ex. \textipa{[""Ekspl@"neIS@n]}

$\Rightarrow$ \url{http://en.wikibooks.org/wiki/LaTeX/Linguistics#IPA_characters}


% \end{frame}

\begin{frame}
\frametitle{Math formulae}
Formulae in texts: $a^2 + b^2 = c^2$

\noindent Formulae in equation environment:
\begin{equation}
a^2 + b^2 = c^2
\end{equation}
$\Rightarrow$ \url{http://en.wikibooks.org/wiki/LaTeX/Mathematics}
\end{frame}

\begin{frame}
\frametitle{Tables}
\begin{tabular}{c|c|c}
\hline
cell 11 & cell 12 & cell 13 \\
\hline
cell 21 & cell 22 & cell 23 \\
\hline
\end{tabular}
\end{frame}
